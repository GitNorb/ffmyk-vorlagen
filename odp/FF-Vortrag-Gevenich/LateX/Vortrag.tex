%\documentclass[aspectratio=43]{beamer}
\documentclass[c]{beamer}
\usetheme{intridea}  %% Themenwahl

\usepackage[ngerman]{babel} 
\usepackage[T1]{fontenc}    % richtige Silbentrennung
\usepackage[utf8]{inputenc} % Umlaute etc.!
\usepackage{eurosym}
\usepackage{tikz}

\usetikzlibrary{arrows,decorations.pathmorphing,backgrounds,fit,positioning,shapes.symbols,chains}

%1

\title{Freifunk-MYK}
\author{Norbert Härig}
\date{\today}

%2
\begin{document}
\maketitle

\begin{frame}{Was ist Freifunk?}
	\begin{itemize}
		\item Initiative für freie, offene, kostenlose Netzwerke
		\item Öffentlich - freifunk steht jedem offen, als Nutzer oder Anbieter
		\item Im Besitz der Gemeinschaft - Wird von den Menschen betrieben, die es nutzen
		\item Nicht kommerziell
		\item Ausschliesslich freie, quelloffene Programme
		\item Netzneutral - keine Manipulation der Datenströme
		\item In  \href{http://freifunk.net/wie-mache-ich-mit/community-finden/}{139 Orten} gibt es bereits Freifunknetze mit mehr als 7500 Zugangspunkten
	\end{itemize}
\end{frame}



\end{document}